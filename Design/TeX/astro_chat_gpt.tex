\documentclass[12pt,a4paper]{article}

\usepackage[utf8]{inputenc}
\usepackage[english]{babel}
\usepackage{amsmath}
\usepackage{amsfonts}
\usepackage{amssymb}
\usepackage{graphicx}
\usepackage{float}
\usepackage{tabularx}
\usepackage{xspace}
\usepackage{dirtree}
\usepackage{url}
\usepackage{color}
\usepackage[toc,page]{appendix}
\usepackage[colorlinks=false, pdfborder={0 0 0}]{hyperref}
\usepackage[left=2.30cm, right=1.50cm, top=2.00cm, bottom=1.80cm]{geometry}

% USAGE DIRTREE
%==============
%\dirtree{%
%	.1 Root.
%	.2 Level 2 \DTcomment{Level 2 comment}.
%	.3 Level 3 \DTcomment{Level 3 comment}.
%}

% USAGE TABULARX
%================
%\begin{tabularx}{0.95\textwidth}{|l|X|}
%	\hline
%	Header 1	& Header 2		\\
%	\hline
%	Line 1 Left	& Line 1 Right 	\\
%	Line 2 Left	& Line 2 Right 	\\
%	\hline
%\end{tabularx}

%USAGE ITEMIZE WITH ITEMSEP
%==========================
%\begin{itemize}\itemsep-4pt
%\item First Item
%\item Second Item
%\item ...
%\item Last Item
%\end{itemize}

% USAGE APPENDICES
%=================
%\appendices
%\section{Important Equations}

% MATRICES
%=========
% matrix		no borders
% pmatrix		()
% bmatrix		[]
% vmatrix		||
% Vmatrix		|| ||
% BMatrix		{}

% ALIGNMENT
% =========
% \begin{equation}\label{key}
% 	\begin{aligned}
% 		a   &= x_{ij} \\
% 		b_j &= y_j 
% 	\end{aligned}
% \end{equation}

% AUTO COUNTER (the prefix here being "Requirement")
%===================================================
\newcounter{ReqCounter}
\newcommand{\req}{\protect\stepcounter{ReqCounter}\textbf{RQ} \textbf{\theReqCounter.}~}

% \input{../../../../Documents/macros}

\newcommand{\pal}{\texttt{palipali}\xspace}

\author{Stanislav Koncebovski, ChatGPT 3.5}
\title{Astro}



\begin{document}
	\maketitle
	
	To transform the ecliptic coordinates of the Sun (longitude and latitude) into equatorial coordinates (right ascension and declination), follow these steps:
	
	\section{Heliocentric Ecliptic System}
		\subsection{Ecliptic Coordinates}
			Defined by the plane of the Earth's orbit around the Sun. The ecliptic longitude (\(\lambda\)) and latitude (\(\beta\)) are the angular measurements along this plane.
		\subsection{Equatorial Coordinates}
			Defined by the Earth's equator projected onto the celestial sphere. The right ascension (\(\alpha\)) and declination (\(\delta\)) are the angular measurements in this system.
	
	\section{Ecliptic Longitude (\(\lambda\)) and Latitude (\(\beta\))}
	%2. **Obtain the Ecliptic Longitude (\(\lambda\)) and Latitude (\(\beta\))**:
		For the Sun, the ecliptic latitude \(\beta\) is approximately 0° because the Sun's apparent path follows the ecliptic plane. Therefore, you mainly need the ecliptic longitude \(\lambda\).
	
	\section{Calculate the Obliquity of the Ecliptic (\(\epsilon\))}
	% 3. ****:
		The obliquity of the ecliptic is the angle between the Earth's equatorial plane and the ecliptic plane. This value changes slightly over time. For many purposes, the mean obliquity can be used:
	\[
	\epsilon \approx 23.4397^\circ
	\]
	
	\section{Convert Ecliptic Coordinates (\(\lambda, \beta\)) to Equatorial Coordinates (\(\alpha, \delta\))}
	% 4. ****:
		Use the following formulas:
	\[
	\tan(\alpha) = \frac{\sin(\lambda) \cos(\epsilon) - \tan(\beta) \sin(\epsilon)}{\cos(\lambda)}
	\]
	\[
	\sin(\delta) = \sin(\beta) \cos(\epsilon) + \cos(\beta) \sin(\epsilon) \sin(\lambda)
	\]
	Given that \(\beta\) is 0 for the Sun, the formulas simplify to:
	\[
	\tan(\alpha) = \frac{\sin(\lambda) \cos(\epsilon)}{\cos(\lambda)}
	\]
	\[
	\sin(\delta) = \sin(\epsilon) \sin(\lambda)
	\]
	
	\section{Calculate Right Ascension (\(\alpha\)) and Declination (\(\delta\))}
	% 5. ****:
	Right Ascension \(\alpha\):
	\[
	\alpha = \arctan \left( \frac{\sin(\lambda) \cos(\epsilon)}{\cos(\lambda)} \right)
	\]
	Ensure that \(\alpha\) is in the correct quadrant.
	Declination \(\delta\):
	\[
	\delta = \arcsin (\sin(\epsilon) \sin(\lambda))
	\]
	
	\section{Example Calculation} 
	
	Let’s consider an example where the ecliptic longitude \(\lambda\) of the Sun is 100°.
	
	\begin{enumerate}
	\item Ecliptic Longitude (\(\lambda\)) \\
		\(\lambda = 100^\circ\)
	\item Obliquity of the Ecliptic (\(\epsilon\)) \\
		\(\epsilon = 23.4397^\circ\)
		
	\item Convert Ecliptic to Equatorial Coordinates \\
		Calculate \(\tan(\alpha)\):
		\[
		\tan(\alpha) = \frac{\sin(100^\circ) \cos(23.4397^\circ)}{\cos(100^\circ)}
		\]
		\[
		\tan(\alpha) = \frac{0.9848 \times 0.9175}{-0.1736} \approx -5.206
		\]
		Therefore,
		\[
		\alpha = \arctan(-5.206)
		\]
		
		Calculate \(\sin(\delta)\):
		\[
		\sin(\delta) = \sin(23.4397^\circ) \times \sin(100^\circ)
		\]
		\[
		\sin(\delta) = 0.3978 \times 0.9848 \approx 0.3917
		\]
		Therefore,
		\[
		\delta = \arcsin(0.3917) \approx 23^\circ
		\]
	\item Last Item
	\end{enumerate}
	
%	1. **Ecliptic Longitude (\(\lambda\))**:
%	- \(\lambda = 100^\circ\)
%	
%	2. **Obliquity of the Ecliptic (\(\epsilon\))**:
%	- \(\epsilon = 23.4397^\circ\)
%	
%	3. **Convert Ecliptic to Equatorial Coordinates**:
%	- Calculate \(\tan(\alpha)\):
%	\[
%	\tan(\alpha) = \frac{\sin(100^\circ) \cos(23.4397^\circ)}{\cos(100^\circ)}
%	\]
%	\[
%	\tan(\alpha) = \frac{0.9848 \times 0.9175}{-0.1736} \approx -5.206
%	\]
%	- Therefore,
%	\[
%	\alpha = \arctan(-5.206)
%	\]
%	
%	- Calculate \(\sin(\delta)\):
%	\[
%	\sin(\delta) = \sin(23.4397^\circ) \times \sin(100^\circ)
%	\]
%	\[
%	\sin(\delta) = 0.3978 \times 0.9848 \approx 0.3917
%	\]
%	- Therefore,
%	\[
%	\delta = \arcsin(0.3917) \approx 23^\circ
%	\]
	
	By performing these calculations, you can transform the ecliptic longitude of the Sun into its corresponding right ascension and declination. In this example, the right ascension and declination are approximately:
	- \(\alpha \approx 258.3^\circ\)
	- \(\delta \approx 23^\circ\)
	
	For a precise calculation, you can use more accurate values for the obliquity of the ecliptic and handle the conversion carefully using a programming language or a dedicated astronomical tool.
	
	The difference between the Sun-centered (heliocentric) and Earth-centered (geocentric) ecliptic longitudes arises due to the relative positions of the Earth and the Sun in their respective frames of reference. When working with the Earth-centered ecliptic longitude \(l\) of the Sun, it needs to be converted to the Sun-centered ecliptic longitude \(\lambda\) before using the transformation formulas for equatorial coordinates.
	
	\section{Geocentric Ecliptic System}
	
	To convert the Earth-centered ecliptic longitude \(l\) of the Sun to the Sun-centered ecliptic longitude \(\lambda\), follow these steps:
	
	\begin{enumerate}
		\item Calculate the Sun's True Anomaly (\(v\)) \\
		The true anomaly is the angle between the Sun's position and the perihelion, as seen from the center of the elliptical orbit. \\
		You can obtain \(v\) using the mean anomaly and the eccentricity of Earth's orbit, but for simplicity, we assume it is given or can be computed.
		
		\item Calculate the Earth's Mean Longitude (\(L\)) \\
		The mean longitude is the angle measured along the ecliptic from the vernal equinox to the mean position of the Earth (or Sun) as it would be seen from the center of its orbit.
		
		\item Calculate the Geocentric Ecliptic Longitude (\(l\)) \\
		The geocentric ecliptic longitude of the Sun can be obtained from the various [empirical] formulas provided by astronomical sources.
		
		\item Convert Geocentric Longitude to Heliocentric Longitude \\
		To convert the geocentric longitude \(l\) to the heliocentric longitude \(\lambda\), use the following relation:
		\[
		\lambda = l + 180^\circ - 2v
		\]
		Here, \(v\) is the true anomaly of the Earth (or equivalently, the Sun's true anomaly as seen from the Earth).
	\end{enumerate}
	
%	1. **Calculate the Sun's True Anomaly (\(v\))**:
%	- The true anomaly is the angle between the Sun's position and the perihelion, as seen from the center of the elliptical orbit.
%	- You can obtain \(v\) using the mean anomaly and the eccentricity of Earth's orbit, but for simplicity, we assume it is given or can be computed.
%	
%	2. **Calculate the Earth's Mean Longitude (\(L\))**:
%	- The mean longitude is the angle measured along the ecliptic from the vernal equinox to the mean position of the Earth (or Sun) as it would be seen from the center of its orbit.
%	
%	3. **Calculate the Geocentric Ecliptic Longitude (\(l\))**:
%	- The geocentric ecliptic longitude of the Sun can be obtained from the various formulas provided by astronomical sources.
%	
%	4. **Convert Geocentric Longitude to Heliocentric Longitude**:
%	- To convert the geocentric longitude \(l\) to the heliocentric longitude \(\lambda\), use the following relation:
%	\[
%	\lambda = l + 180^\circ - 2v
%	\]
%	- Here, \(v\) is the true anomaly of the Earth (or equivalently, the Sun's true anomaly as seen from the Earth).
	
	\section{Example Calculation} 
	
	Assume we have the following values:
	- Geocentric ecliptic longitude of the Sun \(l = 100^\circ\)
	- True anomaly \(v = 10^\circ\)
	
	Now, apply the formula to convert \(l\) to \(\lambda\):
	
	\[
	\lambda = l + 180^\circ - 2v
	\]
	\[
	\lambda = 100^\circ + 180^\circ - 2 \times 10^\circ
	\]
	\[
	\lambda = 100^\circ + 180^\circ - 20^\circ
	\]
	\[
	\lambda = 260^\circ
	\]
	
	\section{Using the Converted Longitude}
	
	Once you have \(\lambda\), you can proceed with the transformation from ecliptic coordinates to equatorial coordinates using the formulas provided earlier.
	
	\section{Full Transformation Workflow}
	
	\begin{enumerate}
		\item Calculate the geocentric ecliptic longitude \(l\)
		\item Calculate the true anomaly \(v\)
		\item Convert \(l\) to \(\lambda\) \\
			\[
				\lambda = l + 180^\circ - 2v
			\]
		\item Transform \(\lambda\) to equatorial coordinates (\(\alpha\), \(\delta\)) \\
			Right Ascension \(\alpha\):
			\[
			\alpha = \arctan \left( \frac{\sin(\lambda) \cos(\epsilon)}{\cos(\lambda)} \right)
			\]
			Declination \(\delta\):
			\[
			\delta = \arcsin (\sin(\epsilon) \sin(\lambda))
			\]
	\end{enumerate}
	
%	1. **Calculate the geocentric ecliptic longitude \(l\)**.
%	2. **Calculate the true anomaly \(v\)**.
%	3. **Convert \(l\) to \(\lambda\)**:
%	\[
%	\lambda = l + 180^\circ - 2v
%	\]
%	4. **Transform \(\lambda\) to equatorial coordinates (\(\alpha\), \(\delta\))**:
%	- Right Ascension \(\alpha\):
%	\[
%	\alpha = \arctan \left( \frac{\sin(\lambda) \cos(\epsilon)}{\cos(\lambda)} \right)
%	\]
%	- Declination \(\delta\):
%	\[
%	\delta = \arcsin (\sin(\epsilon) \sin(\lambda))
%	\]
	
	By following these steps, you can accurately transform the Earth-centered ecliptic longitude of the Sun to equatorial coordinates.
\end{document}